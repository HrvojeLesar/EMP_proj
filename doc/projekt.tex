\documentclass[]{foi}
\usepackage[utf8]{inputenc}
\usepackage{lipsum}

\vrstaRada{\projekt}

\title{Rad s XML u Javi}
\predmet{\predmetEMP}

\author{Hrvoje Lesar}
\spolStudenta{\musko}

\mentor{Sandro Gerić}
\spolMentora{\musko}
\titulaProfesora{Izv. prof. dr. sc.}

\godina{2023}
\mjesec{Svibanj}

\indeks{0016133479}

\smjer{Organizacija poslovnih sustava}

\begin{document}

\maketitle

\tableofcontents

\makeatletter \def\@dotsep{4.5} \makeatother
\pagestyle{plain}

\chapter{Uvod}

\chapter{XML}
XML je standard za označavanje dokumenata koji podržava W3C organizacija.
Kratica XML stoji za Extensible Markup Language. Kroz XML se definira generička
sintaksa korištena za označavanje podataka jednostavnim, za ljude čitljivim oznakama \cite{xml_in_a_nutshell}.
Ovakav format dokumenta omogućuje visoku razinu kostumizacije načina definiranja
i prikaza podataka. Neki od primjera gdje se koristi su vektorska grafika,
elektronička razmjena podataka, serijalizacija objekata (pretvaranje Java objekta u XML dokument),
poziva udaljenih procedura.

\section{Struktura}
Podatci u XML dokumentu su zapisani kao tekst. Podatci su okruženi tekstualnim oznakama
koje opisuju podatak. Osnovna jedinica podatka i oznake naziva se element. U nastavku će
biti definirane ostale vrste čvorova dostupnih u XML sintaksi. XML specifikacija definira
točnu sintaksu koja se u dokumentu mora poštivati, definira kako su elementi orkuženi
oznakama (tagovima), kako izgleda tag, koji nazivi tagova su prihvatljivi, gdje se
atributi nalaze i njihovu sintaksu.

\begin{lstlisting}[caption={Vrste čvorova}]
<?xml version="1.0" encoding="UTF-8"?>
<Hrvatska>
    <!-- Prognoza za datum 20.05.2023. -->
    <DatumTermin>
        <Datum>20.05.2023</Datum>
        <Termin>20</Termin>
    </DatumTermin>
    <Grad autom="0">
        <GradIme>Varaždin</GradIme>
        <Lat> 46.28</Lat>
        <Lon> 16.36</Lon>
        <Podatci>
            <Temp> 19.7</Temp>
            <Vlaga>75</Vlaga>
            <Tlak>1015.2</Tlak>
            <TlakTend>-</TlakTend>
            <VjetarSmjer>NE</VjetarSmjer>
            <VjetarBrzina> 3.5</VjetarBrzina>
            <Vrijeme>pretežno oblačno</Vrijeme>
            <VrijemeZnak>4</VrijemeZnak>
        </Podatci>
    </Grad>
</Hrvatska>
\end{lstlisting}

U isječku je prikazan jedan XML dokument. Dokument prikazuje prognozu vremena
na određeni dan. Kroz dokument možemo vidjeti različite vrste čvorova koje XML
sintaksa podržava tj. možemo definirati 5 vrsta:
\begin{enumerate}
	\item Uputa za obradu (meta element) - u prvoj liniji definirani je meta element koji
	      opisuje dokument. Prikazani element definira verziju XML-a koju prikazuje dokument
	      i vrstu kodiranja teksta. Meta elementi se koriste za opis dokumenta i ne mogu imati
	      sadržavati druge elemente.
	\item Korijenski element - u primjeru je korijenski element prikazan tagom \texttt{Hrvatska}.
	      Pravilni XML dokument mora sadržavati točno jedan korijenski element u kojemu su sadržani
	      svi ostali elementi dokumenta.
	\item Element - XML element s nazivom, atributima i listom elemenata djece. Primjer
	      elementa može biti bilo koji od elemenata definiranih poslije korijenskog. Element
	      \texttt{Grad} je dobar primjer elementa koji sadrži više elemenata djece, sadrži
	      jedan atribut nazvan \texttt{autom} s vrijednošću \texttt{0}.
	\item Tekst - niz znakova između otvarajućeg i zatvarajućeg taga. \texttt{Varaždin} je
	      primjer teksta, vrijednosti koju sadrđi tag \texttt{GradIme}.
	\item Komentar - koristan za ostavljanje napomena ili komentara u dokumentu. Kod parsiranja
	      ovaj element se preskače. U primjeru se komentar nalazi u trećoj liniji, komentar ne može
	      sadržavati druge elemente, sadržaj komentara su podaci komentara.
\end{enumerate}

\section{Pravila}
Pravilno oblikovanje dokumenta je minimalni zahtjev za XML dokument \cite{w3c_rec}.
Dokument koji nije pravilno oblikovan nije XML dokument. Parseri ga ne mogu pročitati i
nije im dozvoljeno popravljati neispravan dokument, ne može pretpostaviti za koju
svrhu je autor namijenio dokument te ispraviti dokument, kao što je moguće kod HTML-a.
Kako bi XML dokument bio pravilno oblikovan mora sadržavati \cite{process_xml}:
\begin{itemize}
	\item točno jedan korijenski element,
	\item svi početni tagovi moraju imati završne tagove,
	\item vrijednosti u atributima moraju biti u navodnicima,
	\item sadržaj mora biti definiran unutar character seta,
	\item tagovi mogu biti ugniježđeni ali se ne smiju preklapati.
\end{itemize}

\chapter{XML u javi}
U ovom poglavlju ćemo proći kroz nekoliko načina učitavanja i parsiranja XML datoteke
u programskom jeziku Java. Prikazati kako su pristupi parsiranju različiti te koje su
prednosti i nedostaci kod korištenja određene vrste parsera.

\section{XSD}
XSD je kratica za XML Schema Definition. XSD se koristi za definiranje strukture XML
dokumenta, sama shema je pisana u XML-u.
Svi podaci koji se u java projektu učitavaju prvo se validiraju prema XSD zadanoj shemi.
U projektu su korišteni tri različiti XML dokumenti što znači da ujedno postoje i
tri različite XSD sheme.

\begin{lstlisting}[caption={Primjer XSD dokumenta za sedmodnevnu prognozu vremena}]
<xs:schema xmlns:xs="http://www.w3.org/2001/XMLSchema">

    <xs:complexType name="grad">
        <xs:sequence>
            <xs:element name="GradIme" type="xs:string" />
            <xs:element name="Tmax" type="xs:string" />
            <xs:element name="Tmin" type="xs:string" />
            <xs:element name="Tmin5" type="xs:string" />
            <xs:element name="Obor" type="xs:double" />
            <xs:element name="Snijeg" type="xs:string" />
            <xs:element name="VlagaMax" type="xs:integer" />
            <xs:element name="VlagaMin" type="xs:integer" />
            <xs:element name="Sunce" type="xs:string" />
            <xs:element name="Tna5Max" type="xs:string" />
            <xs:element name="Tna5Min" type="xs:string" />
            <xs:element name="Tna20Max" type="xs:string" />
            <xs:element name="Tna20Min" type="xs:string" />
        </xs:sequence>
    </xs:complexType>

    <xs:complexType name="podaci">
        <xs:sequence>
            <xs:element name="Grad" type="grad" minOccurs="0" maxOccurs="unbounded" />
        </xs:sequence>
    </xs:complexType>

    <xs:complexType name="agroMeteoroloskiPodaci">
        <xs:sequence>
            <xs:element name="Naslov" type="xs:string" />
            <xs:element name="Podaci" type="podaci" />
        </xs:sequence>
    </xs:complexType>

    <xs:element name="AgroMeteoroloskiPodaci" type="agroMeteoroloskiPodaci" />

</xs:schema>
\end{lstlisting}

Kroz priloženi dokument je definirana struktura XML dokumenta koji će sadržavati podatke
o sedmodnevnoj prognozi vremena. Shema definira jedan korijenski element koji ima prilagođeni tip podatka.
\texttt{AgroMeteoroloskiPodaci} je korijenski element koji mora biti prisutan u XML dokumentu
koji ova shema validira. Korijenski element mora se sastojati od dva druga elementa, a to su
\texttt{Naslov} i \texttt{Podaci}. \texttt{Naslov} je jednostavni element koji sadrži vrijednost
naslova, dok su \texttt{Podaci} element koji se sastoji od više podelemenata tipa \texttt{Grad}.

\begin{lstlisting}[language=java, caption={Klasa za validaciju XML datoteke prema XSD shemi}]
public class XSDValidator {
    public static void validiraj(File xmlDat, File xsdDat) throws SAXException, IOException {
        SchemaFactory schemaFactory = SchemaFactory.newInstance(XMLConstants.W3C_XML_SCHEMA_NS_URI);
        Schema schema = schemaFactory.newSchema(xsdDat);
        Validator validator = schema.newValidator();
        validator.validate(new StreamSource(xmlDat));
    }
}
\end{lstlisting}

Klasa sadrži jednu metodu koja kao ulaz prima XML datoteku i XSD datoteku.
U metodi se kreira nova instanca \texttt{SchemaFactory} koja kao arugment prima putanju
do sheme koja će se koristiti za validaciju XML dokumenta. U ovom slučaju se koristi
ugrađena konstana \texttt{XMLConstants$.$W3C\_XML\_SCHEMA\_NS\_URI} koja je zapravo
putanja na imenski prostor (namespace) XSD sheme koji definira pravila sheme.
Potom se kreira nova instanca sheme koja za ulazni argument uzima datoteku
sheme, poslijednje se kreira objekt validator kojemu se predaje XML datoteka koju
želimo validirati prema zadanoj shemi. U slučaju da nije moguće validirati ulaznu
datoteku validator postavlja iznimku i valja grešku kod validiranja dokumenta.

\section{SAX}
SAX je prva vrsta parsera koju ćemo pregledati. SAX prasira dokument redak po redak
te pritom nailazka na otvarajući, zatvarajući tag ili podatke pokreće događaj (event).
Prednost SAX-a je što kroz parsiranje koristi minimalno memorije tj. ne učitava cijelu
XML datoteku u memoriju nego inkrementalno prolazi kroz datuteku, te korisnik po potrebi
izvlači podatke koje treba iz datoteke. Tako da je ovaj parser vrlo pogodan za prasiranje
velikih XML dokumenata.

Kako bi mogli parsirati dokument korištenjem ovog parsera prvo je potrebno napraviti
klasu koja nasljeđuje klasu \texttt{DefaultHandler} iz SAX knjižnice. Kako bi parser
mogao pokretati događaje potrebno ih je implementirati u klasi. Neki od najvažnijih
metoda koje se mogu implementirati su \texttt{startElement(...)}, \texttt{characters(...)},
\texttt{endElement(...)}. \texttt{startDocument()} i \texttt{endDocument()}.

\begin{lstlisting}[language=java, caption={Primjer implementacije metoda za prihvačanje događaja}]
public class SAXPrognoza extends DefaultHandler {
    @Override
    public void startElement(String uri, String localName, String qName, Attributes attributes) {
        switch (qName) {
            case "izmjena": {
                this.izmjena = new Izmjena();
                this.izmjena.attrRun = Integer.parseInt(attributes.getValue("run"));
                break;
            }
            case "grad": {
                this.trenutniGrad = new Grad();
                this.trenutniGrad.attrIme = attributes.getValue("ime");
                this.trenutniGrad.attrCode = attributes.getValue("code");
                break;
            }
            case "dan": {
                this.trenutniDan = new Dan();
                this.trenutniDan.attrDatum = attributes.getValue("datum");
                this.trenutniDan.attrDtj = attributes.getValue("dtj");
                this.trenutniDan.attrSat = Integer.parseInt(attributes.getValue("sat"));
                break;
            }
        }
    }

    @Override
    public void characters(char[] ch, int start, int lenght) {
        this.trenutnaVrijednost = new String(ch, start, lenght);
    }

    @Override
    public void endElement(String uri, String localName, String qName) {
        switch (qName) {
            case "izmjena": {
                this.izmjena.vrijednost = this.trenutnaVrijednost;
                break;
            }
            case "t_2m": {
                this.trenutniDan.t2m = Integer.parseInt(this.trenutnaVrijednost);
                break;
            }
            case "simbol": {
                this.trenutniDan.simbol = this.trenutnaVrijednost;
                break;
            }
            case "vjetar": {
                this.trenutniDan.vjetar = this.trenutnaVrijednost;
                break;
            }
            case "oborina": {
                this.trenutniDan.oborina = Double.parseDouble(this.trenutnaVrijednost);
                break;
            }
            case "dan": {
                this.trenutniGrad.dan.add(this.trenutniDan);
                break;
            }
            case "grad": {
                gradovi.add(this.trenutniGrad);
                break;
            }
        }
    }
}
\end{lstlisting}

Isječak prikazuje metode koje su implementirane u klasi \texttt{SAXPrognoza} koja se u
projektu koristi za parsiranje sedmodnevne vremenske prognoze. U isječku nije prikazana
potpuna klasa, fale neki članovi klase, no istaknute su metode korištene za parsiranje.
Metoda \texttt{startElement} se poziva kad SAX parser naiđe na početni XML element. Ovisno
o nazivu elementa kreira se novi objekt \texttt{Izmjena}, \texttt{Grad} ili \texttt{Dan}. Istovremeno se za svaki
element su parsirani atributi te dodani objektu, ako element sadrži neke atribute. Metoda
\texttt{characters} se poziva kada parser uspejšno parsira vrijednost između dva elementa.
U klasi vrijednost spremamo za kasnije korištenje. Zadnja korištena metoda je \texttt{endElement}
koja se poziva dok parser naiđe završni element. U metodi se prema nazivu završnog elementa
zapisuje vrijednost tog elementa u odgovarajući objekt. Vrijednost se po potrebi pretvara
u tip podataka koji objekt traži.

Sad kad imamo definiran način kreiranja objekata iz XML dokumenta možemo kreirati instancu
SAX parsera i predati mu dokument koji će parsirati i pretvoriti u objekte koje je moguće
koristiti u programu.

\begin{lstlisting}[language=java, caption={Kreiranje instance SAX parsera i parsiranje XML datoteke}]
SAXParserFactory saxFactory = SAXParserFactory.newInstance();
SAXParser saxParser = saxFactory.newSAXParser();
SAXPrognoza prognoza = new SAXPrognoza();
saxParser.parse(DemoDatoteke.demoTrodnevnaPrognozaXML(), prognoza);
\end{lstlisting}

U ovom isječku je vidljivo kak se kreira instancira SAX parser. Najzanimljiviji dio
je u četvrtoj liniji. Metoda \texttt{parse} kao argumente prima XML datoteku i klasu
koja ima implementiran \texttt{DefaultHandler}. U ovom slučaju klasa \texttt{SAXPrognoza}
implementira potrebne metode te \texttt{parse} koristi iste kod parsiranja tj. kod
pokretanja događaja.

\section{DOM}
\section{JAXB}

\chapter{Zaključak}

\makebackmatter

\end{document}
